% \VignetteIndexEntry{Classes for record linkage of big data sets}
% !Rnw weave = knitr
%\VignetteEngine{knitr::knitr}
%\VignetteEncoding{UTF-8}




\documentclass[a4paper]{article}\usepackage[]{graphicx}\usepackage[]{color}
% maxwidth is the original width if it is less than linewidth
% otherwise use linewidth (to make sure the graphics do not exceed the margin)
\makeatletter
\def\maxwidth{ %
  \ifdim\Gin@nat@width>\linewidth
    \linewidth
  \else
    \Gin@nat@width
  \fi
}
\makeatother

\definecolor{fgcolor}{rgb}{0.345, 0.345, 0.345}
\newcommand{\hlnum}[1]{\textcolor[rgb]{0.686,0.059,0.569}{#1}}%
\newcommand{\hlstr}[1]{\textcolor[rgb]{0.192,0.494,0.8}{#1}}%
\newcommand{\hlcom}[1]{\textcolor[rgb]{0.678,0.584,0.686}{\textit{#1}}}%
\newcommand{\hlopt}[1]{\textcolor[rgb]{0,0,0}{#1}}%
\newcommand{\hlstd}[1]{\textcolor[rgb]{0.345,0.345,0.345}{#1}}%
\newcommand{\hlkwa}[1]{\textcolor[rgb]{0.161,0.373,0.58}{\textbf{#1}}}%
\newcommand{\hlkwb}[1]{\textcolor[rgb]{0.69,0.353,0.396}{#1}}%
\newcommand{\hlkwc}[1]{\textcolor[rgb]{0.333,0.667,0.333}{#1}}%
\newcommand{\hlkwd}[1]{\textcolor[rgb]{0.737,0.353,0.396}{\textbf{#1}}}%
\let\hlipl\hlkwb

\usepackage{framed}
\makeatletter
\newenvironment{kframe}{%
 \def\at@end@of@kframe{}%
 \ifinner\ifhmode%
  \def\at@end@of@kframe{\end{minipage}}%
  \begin{minipage}{\columnwidth}%
 \fi\fi%
 \def\FrameCommand##1{\hskip\@totalleftmargin \hskip-\fboxsep
 \colorbox{shadecolor}{##1}\hskip-\fboxsep
     % There is no \\@totalrightmargin, so:
     \hskip-\linewidth \hskip-\@totalleftmargin \hskip\columnwidth}%
 \MakeFramed {\advance\hsize-\width
   \@totalleftmargin\z@ \linewidth\hsize
   \@setminipage}}%
 {\par\unskip\endMakeFramed%
 \at@end@of@kframe}
\makeatother

\definecolor{shadecolor}{rgb}{.97, .97, .97}
\definecolor{messagecolor}{rgb}{0, 0, 0}
\definecolor{warningcolor}{rgb}{1, 0, 1}
\definecolor{errorcolor}{rgb}{1, 0, 0}
\newenvironment{knitrout}{}{} % an empty environment to be redefined in TeX

\usepackage{alltt}

\usepackage[utf8]{inputenc}
\usepackage[T1]{fontenc}
%\DeclareUnicodeCharacter{0008}{}
%\DeclareUnicodeCharacter{0010}{}
%\DeclareUnicodeCharacter{0011}{}
%\DeclareUnicodeCharacter{000B}{}
\IfFileExists{upquote.sty}{\usepackage{upquote}}{}
\begin{document}
%\SweaveOpts{concordance=TRUE}

\title{Classes for record linkage of big data sets}
\author{Andreas Borg, Murat Sariyar}

\maketitle

As of version 0.3, the package RecordLinkage includes extensions to overcome
the problem of high memory consumption that arises when processing a large
number of records (i.e. building record pairs out of $\geq{}1000$ records
without blocking). In versions 0.3\_x, this was achieved by blockwise on-demand
creation of comparison patterns in an embedded SQLite database (through
package \textit{RSQLite}). Package version 0.4 replaces this mechanism by using file-based
data structures from package \textit{ff}. This approach restricts the amount of
data pairs to the available disk space but speeds up execution and facilitates
the implementation of methods that need to process the whole set of record pairs
(e.g. calculation of optimal classification thresholds).

The interface to the big data methods has is compatible to code written
for version 0.3\_x, so users familiar with these can stick to their existing
workflow (unless access to internal structures like object slots is involved).
Therefore, the following text sticks to the vignette already included in versions
before 0.4 and only technical details are changed to reflect the different
implementation.

In order to facilitate a tidier design, S4 classes and methods were used to
implement the extensions. In favor of backward compatibility and development
time, plans of a complete transition to S4 were dismissed. Nevertheless, the
existing functions were joined with their new counterparts, resulting in
methods which dispatch on the new S4 as well as on the existing S3 classes.
This approach combines two advantages: First, existing code using the package
still works, second, the new classes and methods offer (nearly) the same
interface, i.e. the necessary function calls for a linkage task differ only
slightly. An exception is \texttt{getPairs}, whose arguments differ from the
existing version (see man page).


\section{Defining data and comparison parameters}

The existing S3 class \texttt{"RecLinkData"} is supplemented by the S4 classes
\texttt{"RLBigDataLinkage"} and \texttt{"RLBigDataDedup"} for linking two datasets
and deduplication of one dataset respectively. Both share the common abstract
superclass \texttt{"RLBigData"}.

\begin{knitrout}
\definecolor{shadecolor}{rgb}{0.969, 0.969, 0.969}\color{fgcolor}\begin{kframe}
\begin{alltt}
\hlkwd{library}\hlstd{(RecordLinkage)}
\hlkwd{showClass}\hlstd{(}\hlstr{"RLBigData"}\hlstd{)}
\end{alltt}
\begin{verbatim}
## Virtual Class "RLBigData" [package "RecordLinkage"]
## 
## Slots:
##                                                       
## Name:  frequencies    blockFld  excludeFld   strcmpFld
## Class:     numeric        list     numeric     numeric
##                                                       
## Name:    strcmpFun phoneticFld phoneticFun       pairs
## Class:   character     numeric   character        ffdf
##                                                       
## Name:        Wdata    WdataInd           M           U
## Class:   ff_vector   ff_vector   ff_vector   ff_vector
## 
## Known Subclasses: "RLBigDataDedup", "RLBigDataLinkage"
\end{verbatim}
\begin{alltt}
\hlkwd{showClass}\hlstd{(}\hlstr{"RLBigDataDedup"}\hlstd{)}
\end{alltt}
\begin{verbatim}
## Class "RLBigDataDedup" [package "RecordLinkage"]
## 
## Slots:
##                                                       
## Name:         data    identity frequencies    blockFld
## Class:  data.frame      factor     numeric        list
##                                                       
## Name:   excludeFld   strcmpFld   strcmpFun phoneticFld
## Class:     numeric     numeric   character     numeric
##                                                       
## Name:  phoneticFun       pairs       Wdata    WdataInd
## Class:   character        ffdf   ff_vector   ff_vector
##                               
## Name:            M           U
## Class:   ff_vector   ff_vector
## 
## Extends: "RLBigData"
\end{verbatim}
\begin{alltt}
\hlkwd{showClass}\hlstd{(}\hlstr{"RLBigDataLinkage"}\hlstd{)}
\end{alltt}
\begin{verbatim}
## Class "RLBigDataLinkage" [package "RecordLinkage"]
## 
## Slots:
##                                                       
## Name:        data1       data2   identity1   identity2
## Class:  data.frame  data.frame      factor      factor
##                                                       
## Name:  frequencies    blockFld  excludeFld   strcmpFld
## Class:     numeric        list     numeric     numeric
##                                                       
## Name:    strcmpFun phoneticFld phoneticFun       pairs
## Class:   character     numeric   character        ffdf
##                                                       
## Name:        Wdata    WdataInd           M           U
## Class:   ff_vector   ff_vector   ff_vector   ff_vector
## 
## Extends: "RLBigData"
\end{verbatim}
\end{kframe}
\end{knitrout}

For the two non-virtual classes, the constructor-like function \texttt{RLBigDataDedup}
and \texttt{RLBigDataLinkage} exist, which correspond
to \texttt{compare.dedup} and \texttt{compare.linkage} for the S3 classes and
share most of their arguments.

The following example shows the basic usage of the constructors, for details
consult their documentation.

\begin{knitrout}
\definecolor{shadecolor}{rgb}{0.969, 0.969, 0.969}\color{fgcolor}\begin{kframe}
\begin{alltt}
\hlcom{# deduplicate with two blocking iterations and string comparison}
\hlkwd{data}\hlstd{(RLdata500)}
\hlkwd{data}\hlstd{(RLdata10000)}
\hlstd{rpairs1} \hlkwb{<-} \hlkwd{RLBigDataDedup}\hlstd{(RLdata500,}
           \hlkwc{identity} \hlstd{= identity.RLdata500,}
           \hlkwc{blockfld} \hlstd{=} \hlkwd{list}\hlstd{(}\hlnum{1}\hlstd{,}\hlnum{3}\hlstd{),} \hlkwc{strcmp} \hlstd{=} \hlnum{1}\hlopt{:}\hlnum{4}\hlstd{)}

\hlcom{# link two datasets with phonetic code}
\hlstd{s1} \hlkwb{<-} \hlnum{471}\hlopt{:}\hlnum{500}
\hlstd{s2} \hlkwb{<-} \hlkwd{sample}\hlstd{(}\hlnum{1}\hlopt{:}\hlnum{10000}\hlstd{,} \hlnum{300}\hlstd{)}
\hlstd{identity2} \hlkwb{<-} \hlkwd{c}\hlstd{(identity.RLdata500[s1],} \hlkwd{rep}\hlstd{(}\hlnum{NaN}\hlstd{,} \hlkwd{length}\hlstd{(s2)))}
\hlstd{dataset} \hlkwb{<-} \hlkwd{rbind}\hlstd{(RLdata500[s1,], RLdata10000[s2,])}
\hlstd{rpairs2} \hlkwb{<-} \hlkwd{RLBigDataLinkage}\hlstd{(RLdata500, dataset,}
           \hlkwc{identity1} \hlstd{= identity.RLdata500,}
           \hlkwc{identity2} \hlstd{= identity2,} \hlkwc{phonetic} \hlstd{=} \hlnum{1}\hlopt{:}\hlnum{4}\hlstd{,}
           \hlkwc{exclude} \hlstd{=} \hlstr{"lname_c2"}\hlstd{)}
\end{alltt}
\end{kframe}
\end{knitrout}

\section{Supervised classification}

The existing function \texttt{classifySupv} was transformed to a S4 method
which handles the old S3 object (\texttt{"RecLinkData"}) as well as the new 
classes.  However, at the moment a classificator can only be trained with
an object of class \texttt{"RecLinkData"}.

\begin{knitrout}
\definecolor{shadecolor}{rgb}{0.969, 0.969, 0.969}\color{fgcolor}\begin{kframe}
\begin{alltt}
\hlstd{train} \hlkwb{<-} \hlkwd{getMinimalTrain}\hlstd{(}\hlkwd{compare.dedup}\hlstd{(RLdata500,}
         \hlkwc{identity} \hlstd{= identity.RLdata500,}
         \hlkwc{blockfld} \hlstd{=} \hlkwd{list}\hlstd{(}\hlnum{1}\hlstd{,}\hlnum{3}\hlstd{)))}
\hlstd{rpairs1} \hlkwb{<-} \hlkwd{RLBigDataDedup}\hlstd{(RLdata500,}
           \hlkwc{identity} \hlstd{= identity.RLdata500)}
\hlstd{classif} \hlkwb{<-} \hlkwd{trainSupv}\hlstd{(train,} \hlstr{"rpart"}\hlstd{,} \hlkwc{minsplit}\hlstd{=}\hlnum{2}\hlstd{)}
\hlstd{result} \hlkwb{<-} \hlkwd{classifySupv}\hlstd{(classif, rpairs1)}
\end{alltt}
\end{kframe}
\end{knitrout}

The result is an object of class \texttt{"RLResult"} which contains the
classification result along with the data object.

\begin{knitrout}
\definecolor{shadecolor}{rgb}{0.969, 0.969, 0.969}\color{fgcolor}\begin{kframe}
\begin{alltt}
\hlkwd{showClass}\hlstd{(}\hlstr{"RLResult"}\hlstd{)}
\end{alltt}
\begin{verbatim}
## Class "RLResult" [package "RecordLinkage"]
## 
## Slots:
##                             
## Name:        data prediction
## Class:  RLBigData  ff_vector
\end{verbatim}
\end{kframe}
\end{knitrout}

A contingency table can be viewed via \texttt{getTable}, various error measures
are calculated by \texttt{getErrorMeasures}.

\begin{knitrout}
\definecolor{shadecolor}{rgb}{0.969, 0.969, 0.969}\color{fgcolor}\begin{kframe}
\begin{alltt}
\hlkwd{getTable}\hlstd{(result)}
\end{alltt}
\begin{verbatim}
##            classification
## true status      N      P      L
##           0 124696      0      4
##           1      2      0     48
\end{verbatim}
\begin{alltt}
\hlkwd{getErrorMeasures}\hlstd{(result)}
\end{alltt}
\begin{verbatim}
## $alpha
## [1] 0.04
## 
## $beta
## [1] 3.207698e-05
## 
## $accuracy
## [1] 0.9999519
## 
## $precision
## [1] 0.9230769
## 
## $sensitivity
## [1] 0.96
## 
## $specificity
## [1] 0.9999679
## 
## $ppv
## [1] 0.9230769
## 
## $npv
## [1] 0.999984
\end{verbatim}
\end{kframe}
\end{knitrout}

\section{Weight-based classification}

As with \texttt{"RecLinkData"} objects, weight-based classification with
\texttt{"RLBigData*"} classes includes weight calculation and classification
based on one or two thresholds, dividing links, non-links and, if desired,
possible links. The following example applies classification with
Epilink (see documentation of \texttt{epiWeights} for details):
\begin{knitrout}
\definecolor{shadecolor}{rgb}{0.969, 0.969, 0.969}\color{fgcolor}\begin{kframe}
\begin{alltt}
\hlstd{rpairs1} \hlkwb{<-} \hlkwd{epiWeights}\hlstd{(rpairs1)}
\hlstd{result} \hlkwb{<-} \hlkwd{epiClassify}\hlstd{(rpairs1,} \hlnum{0.5}\hlstd{)}
\hlkwd{getTable}\hlstd{(result)}
\end{alltt}
\begin{verbatim}
##            classification
## true status      N      P      L
##           0 124699      0      1
##           1      4      0     46
\end{verbatim}
\end{kframe}
\end{knitrout}

\section{Evaluation and results}

In addition to \texttt{getTable} and \texttt{getErrorMeasures},
\texttt{getPairs}, which was redesigned as a versatile S4 method, is an
important tool to inspect data and linkage results. For example, the following
code extracts all links with weights greater or equal than 0.7 from the result
set obtained in the last example:

\begin{knitrout}
\definecolor{shadecolor}{rgb}{0.969, 0.969, 0.969}\color{fgcolor}\begin{kframe}
\begin{alltt}
\hlkwd{getPairs}\hlstd{(result,} \hlkwc{min.weight}\hlstd{=}\hlnum{0.7}\hlstd{,} \hlkwc{filter.link}\hlstd{=}\hlstr{"link"}\hlstd{)}
\end{alltt}
\begin{verbatim}
## ============================================================
##    id fname_c1 fname_c2 lname_c1 lname_c2   by bm bd
## 1 290    HELGA ELFRIEDE   BERGER     <NA> 1989  1 18
## 2 466    HELGA ELFRIEDE   BERGER     <NA> 1989  1 28
## 3                                                   
## 4 313   URSULA   BIRGIT  MUELLRR     <NA> 1940  6 15
## 5 457   URSULA   BIRGIT  MUELLER     <NA> 1940  6 15
## 6                                                   
## 7 467   ULRIKE   NICOLE   BECKRR     <NA> 1982  8  4
## 8 472   ULRIKE   NICOLE   BECKER     <NA> 1982  8  4
## 9                                                   
##   is_match Class    Weight
## 1                         
## 2     TRUE     L 0.7786012
## 3                         
## 4                         
## 5     TRUE     L 0.7293529
## 6                         
## 7                         
## 8     TRUE     L 0.7293529
## 9
\end{verbatim}
\end{kframe}
\end{knitrout}

A frequent use case is to inspect misclassified record pairs; for this
purpose two shortcuts are included that call \texttt{getPairs} with
appropriate arguments:

\begin{knitrout}
\definecolor{shadecolor}{rgb}{0.969, 0.969, 0.969}\color{fgcolor}\begin{kframe}
\begin{alltt}
\hlkwd{getFalsePos}\hlstd{(result)}
\end{alltt}
\begin{verbatim}
## ============================================================
##    id fname_c1 fname_c2 lname_c1 lname_c2   by bm bd
## 1 388   ANDREA     <NA>    WEBER     <NA> 1945  5 20
## 2 408   ANDREA     <NA>  SCHMIDT     <NA> 1945  2 20
## 3                                                   
##   is_match Class    Weight
## 1                         
## 2    FALSE     L 0.5067013
## 3
\end{verbatim}
\begin{alltt}
\hlkwd{getFalseNeg}\hlstd{(result)}
\end{alltt}
\begin{verbatim}
## ============================================================
##     id fname_c1 fname_c2 lname_c1 lname_c2   by bm bd
## 1  353     INGE     <NA>   SEIDEL     <NA> 1949  9  4
## 2  355    INGEU     <NA>   SEIDEL     <NA> 1949  8  4
## 3                                                    
## 4  285    ERIKA     <NA>    WEBER     <NA> 1995  2  1
## 5  379    ERIKA     <NA>    WEBER     <NA> 1992  2 29
## 6                                                    
## 7  127     KARL     <NA>    KLEIN     <NA> 2002  6 20
## 8  142     KARL     <NA>   KLEIBN     <NA> 2002  6 29
## 9                                                    
## 10  37 HARTMHUT     <NA> HOFFMSNN     <NA> 1929 12 29
## 11  72  HARTMUT     <NA> HOFFMANN     <NA> 1929 12 29
## 12                                                   
##    is_match Class    Weight
## 1                          
## 2      TRUE     N 0.4948059
## 3                          
## 4                          
## 5      TRUE     N 0.4782410
## 6                          
## 7                          
## 8      TRUE     N 0.4692532
## 9                          
## 10                         
## 11     TRUE     N 0.4081096
## 12
\end{verbatim}
\end{kframe}
\end{knitrout}

\end{document}
